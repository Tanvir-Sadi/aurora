\documentclass[11pt]{article}
\usepackage[margin=1in]{geometry}
\usepackage[T1]{fontenc}
\usepackage[utf8]{inputenc}
\usepackage{longtable}
\usepackage{enumitem}
\usepackage{graphicx}
\graphicspath{{./}{uml/}{latex/sdlc/uml/}}
\usepackage[hidelinks]{hyperref}

\title{Software Design Specification (SDS): Aurora}
\author{Project Team}
\date{2026-02-18}

\begin{document}
\maketitle
\tableofcontents
\newpage

\section{Architecture Overview}
\subsection{Design Style}
Client-side modular design using vanilla JavaScript modules and CSS layers.

\subsection{High-Level Components}
\begin{longtable}{|p{0.24\linewidth}|p{0.66\linewidth}|}
\hline
Component & Responsibility \\
\hline
UI Layer (HTML/CSS) & Layout, button presentation, scene layers, responsive behavior \\
\hline
Interaction Controller (JS) & Event handling, state transitions (default -> night) \\
\hline
Animation Engine (JS) & requestAnimationFrame loop, aurora motion calculations \\
\hline
Accessibility Handler (JS/CSS) & Keyboard support and reduced-motion logic \\
\hline
\end{longtable}

\section{Detailed Design}
\subsection{State Model}
\begin{itemize}[leftmargin=*]
  \item State 1: \texttt{default}
  \item State 2: \texttt{night-active}
  \item Optional state flags: \texttt{reduced-motion}, \texttt{is-animating}
\end{itemize}

\subsection{Event Flow}
\begin{enumerate}[leftmargin=*]
  \item User triggers primary button.
  \item Controller sets \texttt{night-active} class.
  \item Animation engine initializes and starts frame loop.
  \item If reduced-motion is active, engine runs reduced effect path.
\end{enumerate}

\subsection{File-Level Design (Proposed)}
\begin{itemize}[leftmargin=*]
  \item \texttt{index.html}: structural markup and control element.
  \item \texttt{styles.css}: scene styling and transitions.
  \item \texttt{app.js}: initialization and interaction control.
  \item \texttt{aurora.js}: animation loop and rendering logic.
\end{itemize}

\section{Design Decisions}
\begin{longtable}{|p{0.18\linewidth}|p{0.72\linewidth}|}
\hline
Decision & Rationale \\
\hline
Use vanilla JS in v1 & Keeps bundle small and code transparent for learning and SDLC traceability \\
\hline
Use CSS transitions for base mode switch & Efficient and simple for broad browser support \\
\hline
Use requestAnimationFrame & Browser-native animation timing and control \\
\hline
Include reduced-motion branch & Accessibility and performance fallback \\
\hline
\end{longtable}

\section{UML Design Diagrams (PlantUML)}
\subsection{Component Diagram}
\begin{center}
  \includegraphics[width=0.52\linewidth]{sds_component.png}\\
  \small Figure: Aurora Component Diagram
\end{center}
\noindent PlantUML source: \texttt{uml/sds\_component.puml}

\subsection{State Diagram}
\begin{center}
  \includegraphics[width=0.62\linewidth]{sds_state.png}\\
  \small Figure: Aurora UI State Diagram
\end{center}
\noindent PlantUML source: \texttt{uml/sds\_state.puml}

\subsection{Sequence Diagram (Activation Flow)}
\begin{center}
  \includegraphics[width=0.88\linewidth]{sds_sequence.png}\\
  \small Figure: Aurora Activation Sequence
\end{center}
\noindent PlantUML source: \texttt{uml/sds\_sequence.puml}

\section{Requirement Coverage}
SDS supports FR-01 to FR-06 and NFR-01 to NFR-04 from SRS.

\end{document}
