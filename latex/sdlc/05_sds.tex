\documentclass[11pt]{article}
\usepackage[margin=1in]{geometry}
\usepackage[T1]{fontenc}
\usepackage[utf8]{inputenc}
\usepackage{longtable}
\usepackage{enumitem}
\usepackage{graphicx}
\graphicspath{{./}{uml/}{latex/sdlc/uml/}}
\usepackage[hidelinks]{hyperref}

\title{Software Design Specification (SDS): Aurora}
\author{Project Team}
\date{2026-02-18}

\begin{document}
\maketitle
\tableofcontents
\newpage

\section{Architecture Overview}
\subsection{Design Style}
Client-side modular design using vanilla JavaScript modules and CSS layers.

\subsection{High-Level Components}
\begin{longtable}{|p{0.24\linewidth}|p{0.66\linewidth}|}
\hline
Component & Responsibility \\
\hline
UI Layer (HTML/CSS) & Layout, button presentation, scene layers, responsive behavior \\
\hline
Interaction Controller (JS) & Event handling, state transitions (default -> night) \\
\hline
Animation Engine (JS) & requestAnimationFrame loop, aurora motion calculations and fallback rendering \\
\hline
Aurora Media Layer (Video) & Video-first realistic aurora playback in night mode \\
\hline
Accessibility Handler (JS/CSS) & Keyboard support and reduced-motion logic \\
\hline
\end{longtable}

\section{Detailed Design}
\subsection{State Model}
\begin{itemize}[leftmargin=*]
  \item State 1: \texttt{default}
  \item State 2: \texttt{night-active}
  \item Optional state flags: \texttt{reduced-motion}, \texttt{is-animating}
\end{itemize}

\subsection{Event Flow}
\begin{enumerate}[leftmargin=*]
  \item User triggers primary button.
  \item Controller sets \texttt{night-active} class.
  \item Animation engine initializes and starts frame loop.
  \item Pointer movement updates interactive parallax offsets for scene depth cues.
  \item If reduced-motion is active, engine runs reduced effect path.
\end{enumerate}

\subsection{Aurora Rendering Strategy (v1.2 Realism Update)}
\begin{itemize}[leftmargin=*]
  \item Runtime uses video-first realistic aurora playback for cinematic output with low CPU usage.
  \item Canvas renderer remains as automatic fallback when video asset is unavailable or playback is blocked.
  \item Renderer uses multi-layer curtain geometry with noise-modulated crest lines instead of only sinusoidal bands.
  \item Each layer computes vertical light-veil tails to simulate altitude variation and depth separation.
  \item Atmospheric haze pass is rendered per frame to soften transitions and increase scene depth.
  \item Color pipeline uses dynamic hue drift with gradient falloff and top-edge glow for realistic blending.
  \item UI includes lightweight micro-interactions (button tilt and ripple feedback) to improve perceived responsiveness.
  \item Reduced-motion mode keeps the same architecture with fewer layers and lower motion amplitude.
\end{itemize}

\subsection{File-Level Design (Current)}
\begin{itemize}[leftmargin=*]
  \item \texttt{index.html}: structural markup and control element.
  \item \texttt{styles.css}: scene styling and transitions.
  \item \texttt{app.js}: initialization, interaction control, pointer parallax, and button micro-animations.
  \item \texttt{aurora.js}: fallback canvas renderer with noise functions, curtain generation, haze pass, and frame loop.
  \item \texttt{assets/aurora-loop.webm|mp4}: video-first realistic aurora source.
\end{itemize}

\section{Design Decisions}
\begin{longtable}{|p{0.18\linewidth}|p{0.72\linewidth}|}
\hline
Decision & Rationale \\
\hline
Use video-first realistic mode with canvas fallback & Delivers cinematic realism while keeping runtime lightweight and resilient \\
\hline
Use vanilla JS in v1/v1.2 & Keeps fallback bundle transparent for learning and SDLC traceability \\
\hline
Use CSS transitions for base mode switch & Efficient and simple for broad browser support \\
\hline
Use requestAnimationFrame & Browser-native animation timing and control \\
\hline
Use noise-modulated curtains in v1.2 & Produces less repetitive and more realistic aurora motion patterns \\
\hline
Add atmospheric haze and glow falloff & Improves depth perception and color realism without external libraries \\
\hline
Include reduced-motion branch & Accessibility and performance fallback \\
\hline
\end{longtable}

\section{UML Design Diagrams (PlantUML)}
\subsection{Component Diagram}
\begin{center}
  \includegraphics[width=0.52\linewidth]{sds_component.png}\\
  \small Figure: Aurora Component Diagram
\end{center}
\noindent PlantUML source: \texttt{uml/sds\_component.puml}

\subsection{State Diagram}
\begin{center}
  \includegraphics[width=0.62\linewidth]{sds_state.png}\\
  \small Figure: Aurora UI State Diagram
\end{center}
\noindent PlantUML source: \texttt{uml/sds\_state.puml}

\subsection{Sequence Diagram (Activation Flow)}
\begin{center}
  \includegraphics[width=0.88\linewidth]{sds_sequence.png}\\
  \small Figure: Aurora Activation Sequence
\end{center}
\noindent PlantUML source: \texttt{uml/sds\_sequence.puml}

\section{Requirement Coverage}
SDS supports FR-01 to FR-06 and NFR-01 to NFR-04 from SRS.

\end{document}



