% !TeX program = xelatex
\documentclass[11pt]{article}
\usepackage[margin=1in]{geometry}
\usepackage{fontspec}
\usepackage{xeCJK}
\IfFontExistsTF{Yu Mincho}{\setCJKmainfont{Yu Mincho}}{\setCJKmainfont{MS Mincho}}
\IfFontExistsTF{Yu Gothic UI}{\setCJKmonofont{Yu Gothic UI}}{\setCJKmonofont{MS Gothic}}
\usepackage{longtable}
\usepackage{enumitem}
\usepackage[hidelinks]{hyperref}

\title{プロジェクト憲章: Aurora Interactive Web サイト}
\author{Prepared by Tanvir Sadi}
\date{2026-02-18}

\begin{document}
\maketitle
\tableofcontents
\newpage

\section{プロジェクト概要}
\subsection{プロジェクトの基本}
\begin{itemize}[leftmargin=*]
  \item プロジェクト名: Aurora Interactive Web サイト
  \item 憲章バージョン: 1.1
  \item プロジェクトオーナー: タンヴィル・サディ
  \item 現在のフェーズ: 計画と設計のベースライン
  \item ターゲット MVP リリース: 2026-03-20
\end{itemize}

\subsection{背景}
このプロジェクトは、単一ページのアニメーション Web エクスペリエンスを提供します。ユーザーがボタン 1 つをクリックすると、ページが夜空に切り替わり、オーロラのアニメーションが開始されます。

\subsection{問題提起}
ほとんどの静的ページでは、ユーザー エンゲージメントが制限されています。このプロジェクトには、フロントエンド専用の MVP として提供できる、軽量でアクセスしやすく、パフォーマンスの高いアニメーション エクスペリエンスが必要です。

\subsection{ビジョンステートメント}
デスクトップでもモバイルでも没入感のある、高速でアクセスしやすいワンクリックの Aurora Web サイトを提供します。

\section{現状と決定履歴}
\begin{itemize}[leftmargin=*]
  \item SDLC ドキュメント構造は \texttt{latex/sdlc} に作成されます。
  \item プロジェクトの方向性はフロントエンドのみの MVP として確認されています。
  \item v1 のテクノロジ決定が最終決定されました: \textbf{Vanilla HTML、CSS、および JavaScript}。
  \item パフォーマンスまたは品質の目標が満たされない限り、v1 にはアニメーション ライブラリは必要ありません。
\end{itemize}

\section{目標と達成基準}
\subsection{目的}
\begin{enumerate}[leftmargin=*]
  \item ナイトモードとオーロラアニメーションのワンクリックアクティベーションを実装します。
  \item デスクトップとモバイルに応答性の高い UI を提供します。
  \item アクセシビリティとパフォーマンスのベースライン目標を達成します。
  \item SDLC ドキュメントを実装と同期させます。
\end{enumerate}

\subsection{成功指標}
\begin{longtable}{|p{0.30\linewidth}|p{0.20\linewidth}|p{0.40\linewidth}|}
\hline
メトリック & ターゲット & 測定 \\
\hline
パフォーマンススコア & 灯台 >= 85 & Lighthouse は本番ビルドで実行されます \\
\hline
アクセシビリティスコア & 灯台 >= 90 & キーボードチェックとLighthouse \\
\hline
インタラクションのレイテンシー & <= 200 ミリ秒 & クリックから移行開始までの DevTools のタイミング \\
\hline
アニメーションの滑らかさ & デスクトップでは 55 FPS 以上、モバイルでは 30 FPS 以上 & ブラウザのパフォーマンスプロファイル \\
\hline
\end{longtable}

\section{範囲}
\subsection{範囲内}
\begin{itemize}[leftmargin=*]
  \item 主要なアクション ボタンが 1 つある単一ページ レイアウト。
  \item 夜空の背景の遷移。
  \item アクティブ モードのオーロラ アニメーション。
  \item 動作を減らしたフォールバック動作。
  \item 一般的なモバイルおよびデスクトップ サイズの応答動作。
\end{itemize}

\subsection{範囲外}
\begin{itemize}[leftmargin=*]
  \item バックエンド API とユーザー アカウント。
  \item CMSの機能。
  \item ネイティブモバイルアプリ。
  \item リアルタイムの地理位置情報または天候に基づくオーロラ モデル。
\end{itemize}

\section{制約と仮定}
\begin{itemize}[leftmargin=*]
  \item 予算は最小限で、プロジェクトは個人で所有されます。
  \item 対象ブラウザ:最新のChrome、Edge、Firefox。
  \item デプロイメントは静的ホスティングになります。
  \item 最新のブラウザ機能が利用可能です。
\end{itemize}

\section{マイルストーン}
\begin{longtable}{|p{0.34\linewidth}|p{0.18\linewidth}|p{0.38\linewidth}|}
\hline
マイルストーン & 日付 & 終了基準 \\
\hline
要件と設計ベースラインが完了 & 2026-02-24 & PRD、SRS、SDS ファーストパス承認済み \\
\hline
コアの実装が完了しました & 2026-03-07 & ナイトモードとオーロラ動作を実装 \\
\hline
テストと品質ゲートが完了しました & 2026-03-14 & テスト計画/ケースが実行され、RTM が更新されました \\
\hline
MVPリリース完了 & 2026-03-20 & 導入およびリリースのチェックリストに合格しました \\
\hline
\end{longtable}

\section{初期リスク}
\begin{itemize}[leftmargin=*]
  \item ローエンドデバイスではパフォーマンスが低下します。
  \item ブラウザのレンダリングの違い。
  \item アニメーション調整によるタイムラインのずれ。
\end{itemize}

\section{承認}
\begin{itemize}[leftmargin=*]
  \item 製品所有者: Tanvir Sadi (2026-02-18 承認)
  \item レビュアー:未定
\end{itemize}

\end{document}
