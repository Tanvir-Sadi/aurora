% !TeX program = xelatex
\documentclass[11pt]{article}
\usepackage[margin=1in]{geometry}
\usepackage{fontspec}
\usepackage{xeCJK}
\IfFontExistsTF{Yu Mincho}{\setCJKmainfont{Yu Mincho}}{\setCJKmainfont{MS Mincho}}
\IfFontExistsTF{Yu Gothic UI}{\setCJKmonofont{Yu Gothic UI}}{\setCJKmonofont{MS Gothic}}
\usepackage{longtable}
\usepackage{enumitem}
\usepackage{array}
\usepackage{xurl}
\usepackage[hidelinks]{hyperref}

\title{実装後レビュー (PIR): Aurora}
\author{Project Team}
\date{2026-02-18}

\begin{document}
\maketitle
\tableofcontents
\newpage

\section{レビュー状況}
\begin{itemize}[leftmargin=*]
  \item リリース状態: リリース済み。
  \item リリースバージョン: \texttt{v1.1.0}。
  \item リリース日: 2026-02-18。
\end{itemize}

\section{レビュー入力資料}
\begin{itemize}[leftmargin=*]
  \item RTM 証跡により、ベースライン リリースにおける FR-01 から FR-06 および NFR-01 から NFR-04 の合格を確認。
  \item \texttt{07\_test\_cases.tex} における TC-01 から TC-10 の実行ログ。
  \item QA で取得した Lighthouse 証跡および reduced-motion 検証結果。
\end{itemize}

\section{目標と結果}
\begin{itemize}[leftmargin=*]
  \item 目標: Vanilla JS により、夜空モード切り替えとオーロラ演出のインタラクティブ体験を提供する。
  \item 結果: MVP リリース \texttt{v1.1.0} で達成。
  \item 目標: ベースライン SDLC 文書とトレーサビリティを整備する。
  \item 結果: 計画、SRS、SDS、テスト、RTM、リリース文書の同期により達成。
\end{itemize}

\section{うまくいったこと}
\begin{itemize}[leftmargin=*]
  \item 要件トレーサビリティは、SRS から RTM まで一貫性を維持できた。
  \item アクセシビリティ要件と reduced-motion 挙動を実装し、リリース前に検証できた。
  \item 欠陥 D-01 (キーボード有効化の不整合) を本番公開前にクローズできた。
\end{itemize}

\section{課題}
\begin{itemize}[leftmargin=*]
  \item キーボード有効化の挙動に対して、テスト期間中に修正サイクルが必要だった。
  \item リリース準備状況の追跡で、文書間の手動同期作業が複数発生した。
\end{itemize}

\section{学んだ教訓}
\begin{itemize}[leftmargin=*]
  \item 実装スプリント早期に、キーボード操作と reduced-motion の確認を実施する。
  \item テスト証跡の更新はコード変更と同時にまとめて行い、ドリフトを抑制する。
  \item タグ公開直後に、リリースチェックリストの結果を確実に記録する。
\end{itemize}

\section{アクションアイテム}
\small
\begin{longtable}{|>{\raggedright\arraybackslash}p{0.08\linewidth}|>{\raggedright\arraybackslash}p{0.26\linewidth}|>{\raggedright\arraybackslash}p{0.11\linewidth}|>{\raggedright\arraybackslash}p{0.11\linewidth}|>{\raggedright\arraybackslash}p{0.14\linewidth}|>{\raggedright\arraybackslash}p{0.22\linewidth}|}
\hline
ID & アクションアイテム & 所有者 & 期日 & 状態 & 証拠 \\
\hline
A-01 & 次回リリース サイクル向けに、軽量なスクリプト式リリースノート テンプレートを追加する & Tanvir & 2026-03-10 & 完了 (2026-02-18) & \url{RELEASE_NOTES_TEMPLATE.md}; \url{scripts/new-release-notes.ps1} \\
\hline
A-02 & リアルな Aurora 目標、UX 改善範囲、性能ガードレールを含む v1.2 バックログを定義する & Tanvir & 2026-03-10 & 完了 (2026-02-18) & \url{latex/sdlc/12_v1_2_backlog.tex} \\
\hline
\end{longtable}
\normalsize

\section{フォローアップ状況 (v1.2 プログラム)}
\begin{itemize}[leftmargin=*]
  \item B-01 から B-03 を実装済み (video-first のリアリズム実行系 + fallback)。
  \item B-05 を実装済み (視差、カーソルグロー、ボタン フィードバック)。
  \item TC-11 から TC-17 の検証は WP-09 として保留中。
  \item 次回 PIR 更新は、\texttt{v1.2.0} のリリース判断後に発行する。
\end{itemize}

\end{document}
