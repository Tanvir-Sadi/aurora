% !TeX program = xelatex
\documentclass[11pt]{article}
\usepackage[margin=1in]{geometry}
\usepackage{fontspec}
\usepackage{xeCJK}
\IfFontExistsTF{Yu Mincho}{\setCJKmainfont{Yu Mincho}}{\setCJKmainfont{MS Mincho}}
\IfFontExistsTF{Yu Gothic UI}{\setCJKmonofont{Yu Gothic UI}}{\setCJKmonofont{MS Gothic}}
\usepackage{longtable}
\usepackage{enumitem}
\usepackage{graphicx}
\graphicspath{{./}{../sdlc/uml/}{latex/sdlc/uml/}}
\usepackage[hidelinks]{hyperref}

\title{ソフトウェア要件仕様 (SRS): Aurora}
\author{Project Team}
\date{2026-02-18}

\begin{document}
\maketitle
\tableofcontents
\newpage

\section{導入}
\subsection{目的}
バニラ HTML、CSS、JavaScript を使用して実装された Aurora MVP のソフトウェア要件を定義します。

\subsection{範囲}
このシステムは、インタラクティブなビジュアル アニメーション トリガーを備えたフロントエンド専用のシングルページ アプリケーションです。

\section{全体的な説明}
\subsection{製品の視点}
Aurora はブラウザ内で実行され、v1 のバックエンド統合は必要ありません。

\subsection{動作環境}
\begin{itemize}[leftmargin=*]
  \item デスクトップブラウザ: 最新の Chrome、Edge、Firefox。
  \item モバイル ブラウザ: 最新の Chromium ベースおよび Safari 互換エンジン。
\end{itemize}

\section{特定の要件}
\subsection{機能要件}
\begin{longtable}{|p{0.14\linewidth}|p{0.72\linewidth}|}
\hline
ID & 声明 \\
\hline
FR-01 & システムは、ページの初期状態で主アクション ボタンを表示します。 \\
\hline
FR-02 & ボタンがアクティブになると、システムはページの背景を夜空モードに切り替えます。 \\
\hline
FR-03 & システムは、ナイトモードのアクティブ化後にオーロラアニメーションを開始します。 \\
\hline
FR-04 & システムは、主要なアクション ボタンのキーボードのアクティブ化を許可します。 \\
\hline
FR-05 & システムは、ターゲット ビューポート サイズ全体でコンテンツを使用可能な状態に保ちます。 \\
\hline
FR-06 & モーションの低減が要求された場合、システムはアニメーションの強度を低減します。 \\
\hline
\end{longtable}

\subsection{非機能要件}
\begin{longtable}{|p{0.14\linewidth}|p{0.72\linewidth}|}
\hline
ID & 声明 \\
\hline
NFR-01 & 最初の相互作用は、ベースライン条件下で 200 ミリ秒以内に開始されます。 \\
\hline
NFR-02 & ページには、ベースライン チェックで重大なアクセシビリティ違反があってはなりません。 \\
\hline
NFR-03 & システムは、v1 の重いアニメーション ライブラリへの依存を回避します。 \\
\hline
NFR-04 & このページは、運用環境での使用フローで捕捉されないランタイム エラーを回避する必要があります。 \\
\hline
\end{longtable}

\section{外部インターフェースの要件}
\begin{itemize}[leftmargin=*]
  \item UI: 1 つのプライマリ アクティベーション コントロールおよびビジュアル シーン レイヤー。
  \item ハードウェア: 標準キーボード、マウス、タッチ インターフェイス。
  \item ソフトウェア: JavaScript が有効になっているブラウザ。
\end{itemize}

\section{UML アーティファクト (PlantUML)}
\subsection{ユースケース図}
\begin{center}
  \includegraphics[width=0.88\linewidth]{srs_use_case.png}\\
  \small Figure: Aurora Use Case Diagram
\end{center}
\noindent PlantUML source: \texttt{uml/srs\_use\_case.puml}

\subsection{アクティビティ図 (主要なユーザー フロー)}
\begin{center}
  \includegraphics[width=0.72\linewidth]{srs_activity.png}\\
  \small Figure: Aurora Primary Activity Flow
\end{center}
\noindent PlantUML source: \texttt{uml/srs\_activity.puml}

\section{トレーサビリティに関するメモ}
このドキュメントの要件 ID は、SDS、テスト ケース、および RTM で使用されます。

\end{document}
